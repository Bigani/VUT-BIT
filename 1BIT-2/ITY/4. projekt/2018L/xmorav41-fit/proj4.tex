\documentclass[a4paper, 11pt]{article}
\usepackage[czech]{babel}
\usepackage[utf8]{inputenc}
\usepackage[left=2cm, top=3cm, text={17cm,24cm}]{geometry}
\usepackage[unicode]{hyperref}
\usepackage{times}




\begin{document}

\begin{titlepage}
\begin{center}

\Huge
\textsc{Vysoké učení technické v~Brně}
\\
\huge
\textsc{Fakulta informačních technologií}

\vspace{\stretch{0.382}}

\LARGE {Typografie a~publikování -- 4. projekt}
\\
\Huge {Bibliografické citace}

\vspace{\stretch{0.618}}

\end{center}
\noindent
{\Large \today \hfill Tomáš Moravčík (xmorav41)}
\end{titlepage}


\section{\LaTeX}

\subsection{Úvod}
\LaTeX \ je typografický systém, ktorý má dve zdanlivé nevýhody, ktoré odrádzajú nových používateľov. Prvou je, že užívateľ potrebuje mať isté znalosti o~ňom a druhou je tvorba textu podobná písaniu programu v programovacom jazyku. \cite{sbornik}

\subsection{História}
Okolo roku 1983 profesor Donald E. Knuth vytvoril typografický systém \TeX, ktorý používal pre svoje potreby na sadzbu textov a matematických rovníc, pričom zachoval vysokú úroveň dokumentu. Následne sa~vytvorili mnohé nadstavby, ktoré uľahčili prácu pre bežných užívateľov. Najznámejšou nadstavbou sa~stal \LaTeX, vytvorený 1994 Leslie Lamportom. \cite{thes1}

\subsection{Prečo \LaTeX}
\LaTeX je populárnou voľbou technických publikácii, ako napríklad články, správy, knihy a dizertácie, ktoré sú plné formúl a rovníc. V minulosti kým text bol písaný strojom, formule sa~písali ručne alebo najatím špecialistu. Každý časopis má vlastné formátovanie a vlastný dizajn, ktorý sa~dá prostredníctvom latexu ľahko vytvoriť \cite{jour1}

\subsection{Viacjazyčnosť}
 Vďaka veľkej popularite sa \LaTeX značne rozšíril vo svete a to malo za príčinu jeho rozšírenia pre jazyky nevyužívajúce latinku, ako napríklad ruština alebo čínština. Väčšinu jazykov zabezpečuje systém \texttt{babel}, pričom pre jazyky ako japončina alebo kórejčina vznikli vlastné implementácie. \cite{Companion}

\subsection{Prostredia}
Najčastejšími prostrediami s~ktorými sa študent naučí pracovať sú tie najpoužívanejšie \cite{jour2}:
\begin{itemize}
    \item Theorémové prostredia  
    \item Rovnicové prostredia
    \item Tabulárové prostredia
    \item Textové prostredia
    
\end{itemize}

\subsection{Grafika}
V dnešnej dobe využitie obrázkov a grafických náčrtkov je veľmi bežné za účelom rýchleho sprostredkovania informácii. \LaTeX umožňuje vytvoriť virtuálne akýkoľvek grafický súbor, napríklad kresby pre matematické účely. \cite{Syrop}

\subsection{Preambula dokumentu}
Preambula sa nachádza medzi príkazmi \verb|\documentclass| a \verb|\begin{dokument}|. Jedná sa o časť dokumentu, s~ktorou sa nastavia rôzne vlastnosti daného výstupného dokumentu, prostredníctvom rôznych balíčkov alebo vlastných príkazov. Najzaujímavejšie informácie sú \cite{thes2}:
\begin{itemize}
    \item Znaková sada vstupného dokumentu
    \item Grafický štýl
    \item Titulok, meno autora a dátum vytvorenia
\end{itemize}

\subsection{\textsc{Bib}\TeX}
\textsc{Bib}\TeX \ umožňuje v prípade využitia množstva literatúry, na ktorú sa chcete odkazovať, vytvoriť zoznam použitej literatúry, spravidla zobrazený na konci dokumentu. \cite{online1} 
\par
\textsc{Bib}\TeX \ využíva súbor so~zoznamom referencií o literatúrach, napríklad meno autora, rok vydania, vydavateľstvo a mnoho iných. Súbor má koncovku .bib a neformátuje dané informácie, ale ich iba skladuje. Až \textsc{Bib}\TeX \ formátuje informácie pre výsledný dokument. \cite{online3} 


\subsection{Ako osadzovať česky}
Typografické pravidlá sú odlišné pre češtinu oproti angličtine, preto sa vytvorili modifikácie programov \TeX, \LaTeX \ a \texttt{pdflatex}, ktoré zabezpečujú české osádzanie. Pre češtinu sa \TeX \ volá \texttt{csplain}, český \LaTeX je \texttt{cslatex} a český \texttt{pdflatex} je \texttt{pdfcslatex}, pričom ich ovládanie je rovnaké ako pre anglické verzie. \cite{online2} 


\newpage
\bibliographystyle{czechiso}
\bibliography{proj4}






\end{document}
