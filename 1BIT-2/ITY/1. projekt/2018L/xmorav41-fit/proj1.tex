\documentclass[a4paper, 10pt, twocolumn]{article}
\usepackage{hyperref}
\usepackage[utf8]{inputenc}
\usepackage[czech]{babel}


\usepackage[left=1.5cm,text={18cm,24cm},top=2cm]{geometry} 
\normalsize
\usepackage{blindtext}

\title{Typografie a publikování -- 1. projekt}
\author{Tomáš Moravčík \href{mailto:xmorav41@stud.fit.vutbr.cz}{(xmorav41@stud.fit.vutbr.cz)} }
\date{}

\newenvironment{block}
  {\leftskip2em
   \rightskip
   \leftskip}



\begin{document}

\maketitle

\section{Hladká sazba}
\label{1:section}

Hladká sazba používá jeden stupeň, druh a řez písma a~je sázena na stanovenou šířku odstavce. Skládá se z odstavců, obvykle začínajících zarážkou, nejde-li o první odstavec za nadpisem. Mohou ale být sázeny i bez zarážky~-- \mbox{rozhodující je celková} grafická úprava. Odstavec končí vý-chodovou řádkou. Věty nesmějí začínat číslicí.

Zvýraznění barvou, podtržením, ani změnou písma se v~odstavcích nepoužívá. Hladká sazba je určena především pro delší texty, jako je beletrie. Porušení konzistence sazby působí v textu rušivě a unavuje čtenářův zrak.



\section{Smíšená sazba}
\label{2:section}
\fontdimen2\font=2pt
Smíšená sazba má o něco volnější pravidla.~Klasická~hladká sazba se doplňuje o další řezy písma pro zvýraznění důle-žitých pojmů. Existuje \uv{pravidlo}:\\

\begin{block}

\noindent
Čím více 
\textbf{druhů},
\emph{řezů}, 
{\scriptsize velikostí},
barev písma~a~ji-ných 
{\tiny \textbf{efektů}} 
použijeme, tím 
\emph{profesionálněji} bude dokument vypadat. Čtenář tím 
{\large bude} 
{\huge vždy NADŠEN} 
{\Huge !}\\

\end{block}

\textsc{Tímto pravidlem se nikdy nesmíte řídit.}
 Příliš časté zvýrazňování textových elementů a změny velikosti písma jsou známkou amatérismu autora a působí velmi ru-šivě. Dobře navržený dokument nemá obsahovat více než
4~řezy či druhy písma. Dobře navržený dokument je de-centní, ne chaotický.

Důležitým znakem správně vysázeného dokumentu je konzistence -- například \textbf{tučný řez} písma bude vyhrazen pouze pro klíčová slova, skloněný řez pouze pro doposud neznámé pojmy a nebude se to míchat. Skloněný řez ne-působí tak rušivě a používá se častěji. V \LaTeX{u} jej sázíme raději příkazem \verb|\emph{text}| než \verb|\textit{text}|.

Smíšená sazba se nejčastěji používá pro sazbu vědec-kých článků a technických zpráv. U delších dokumentů vědeckého či technického charakteru je zvykem vysvětlit význam různých typů zvýraznění v úvodní kapitole.



\section{Další rady:}
\label{3:section}
\begin{itemize}
\item Nadpis nesmí končit dvojtečkou a nesmí obsahovat \mbox{odkazy} na obrázky, citace, poznámky pod čarou, ...

\item Nadpisy, číslování a odkazy na číslované elementy musí~být~sázeny příkazy k tomu určenými.

\item Výčet ani obrázek nesmí začínat hned pod nadpisem a~nesmí tvořit celou kapitolu.

\item Poznámky pod čarou\footnote{Příliš mnoho poznámek pod čarou čtenáře zbytečně rozptyluje.} používejte opravdu střídmě. (Šetřete i s textem v závorkách.)

\item Nepoužívejte velké množství malých obrázků. Zvažte, zda je nelze seskupit.

\item Bezchybným pravopisem a sazbou dáváme najevo \mbox{úctu ke} čtenáři. Odbytý text s chybami bude čtenář právem považovat za nedůvěryhodný.  
\end{itemize}

\section{České odlišnosti}
\label{4:section}

Česká sazba se oproti okolnímu světu v některých aspek-tech~mírně liší. Jednou z odlišností je sazba uvozovek. Uvozovky se v češtině používají převážně pro zobrazení přímé řeči, zvýraznění přezdívek a ironie. V češtině se používají \mbox{uvozovky typu} \uv{9966} místo ``anglických'' uvozovek nebo "dvojitých" uvozovek. Lze je sázet připravenými příkazy nebo při použití UTF-8 kódování i přímo.

Ve smíšené sazbě se řez uvozovek řídí řezem prvního \mbox{uvozovaného slova.} Pokud je uvozována celá věta, sází se \mbox{koncová tečka} před uvozovku, pokud se uvozuje slovo nebo část věty, sází se tečka za uvozovku.

Druhou odlišností je pravidlo pro sázení konců řádků. V české sazbě by řádek neměl končit osamocenou jedno\-písmennou předložkou nebo spojkou. Spojkou \uv{a} končit může pouze při sazbě do šířky 25 liter. Abychom \LaTeX{u} zabránili v sázení osamocených předložek, spojujeme je s~následujícím slovem \emph{nezlomitelnou mezerou}. Tu sázíme pomocí znaku \verb|~| (vlnka, tilda). Pro systematické doplnění vlnek~slouží volně šiřitelný program vlna od pana Olšáka\footnote{Viz http://petr.olsak.net/ftp/olsak/vlna/}.

Balíček fontenc slouží ke korektnímu kódovaní znaků \mbox {s diakritikou, aby bylo} možno v textu vyhledávat a kopí-rovat z něj.



\section{Závěr}
\label{5:section}

Tento dokument schválně obsahuje několik typografických prohřešků. Sekce \ref{2:section} a \ref{3:section} obsahují typografické chyby. V kontextu celého textu je jistě snadno najdete. Je dobré znát možnosti \LaTeX{u}, ale je také nutné vědět, kdy je nepoužít.

\end{document}
