\documentclass[a4paper, 11pt, twocolumn]{article}

\usepackage[utf8]{inputenc}
\usepackage[czech]{babel}

\usepackage[left=1.5cm, top=2.5cm, text={18cm,25cm}]{geometry} 

\usepackage{amsmath}
\usepackage{times}
\usepackage{stackrel}
\usepackage{amssymb}
\usepackage{amsthm}
\usepackage{amsfonts}
\usepackage{verbatim}
\usepackage[IL2]{fontenc}


\theoremstyle{definition}
\newtheorem{definicia}{Definice}
\newtheorem{veta}{Věta}






\begin{document}

\begin{titlepage}
\begin{center}

\Huge
\textsc{Fakulta informačních technologií \\ Vysoké učení technické v~Brně}
\vspace{\stretch{0.382}}

\begin{LARGE}

Typografie a~publikování -- 2. projekt
\\
Sazba dokumentů a~matematických výrazů

\end{LARGE}
\vspace{\stretch{0.618}}

\end{center}

{\Large 2019 \hfill Tomáš Moravčík (xmorav41)}

\end{titlepage}

\newpage

\section*{Úvod}
V~této úloze si vyzkoušíme sazbu titulní strany, matematic\-kých~vzorců, prostředí a~dalších textových struktur obvyk\-lých~pro technicky zaměřené texty (například rovnice (\ref{equation1})
nebo Definice \ref{Definice1} na straně \pageref{Definice1}). Pro odkazovaní na vzorce
a~struktury zásadně používáme příkaz \verb|\label| a~\verb|\ref|
případně \verb|\pageref| pokud se chceme odkázat na stranu
výskytu. \par
Na titulní straně je využito sázení nadpisu podle optického středu s~využitím zlatého řezu. Tento postup byl
probírán na přednášce. Dále je použito odřádkování se \mbox{zadanou relativní} velikostí 0.4\,em a 0.3\,em.

\section{Matematický text}
Nejprve se podíváme na sázení matematických symbolů
a~výrazů v plynulém textu včetně sazby definic a~vět~s~vy\-užitím balíku \verb|amsthm|. Rovněž použijeme poznámku pod
čarou s použitím příkazu \verb|\footnote|. Někdy je vhodné
použít konstrukci \verb|\mbox{}|, která říká, že text nemá být
zalomen.


\begin{definicia}
\label{Definice1}
Zásobníkový automat \textsl{(ZA)} \textit{je definován jako
sedmice tvaru $A = (Q, \Sigma, \Gamma, \delta, q_0, Z_0, F)$, kde:} 

\begin{itemize}
    \item $Q$\textit{ je konečná množina} vnitřních (řídicích) stavů,

    \item \textit{$\Sigma$ je konečná} vstupní abeceda,

    \item \textit{$\Gamma$ je konečná} zásobníková abeceda,

    \item \textit{$\delta$ je} přechodová funkce $Q \times (\Sigma \cup \{ \epsilon\}) \times \Gamma \rightarrow 2^{Q\times\Gamma^*}$, 

    \item $q_0 \in Q$ \textit{je} počáteční stav, $Z_0 \in \Gamma$ \textit{je} startovací symbol zásobníku $a F \subseteq Q$ \textit{je množina} koncových stavů.
\end{itemize}
\end{definicia}

Nechť $P = (Q, \Sigma, \Gamma, \delta, q_0, Z_0, F)$ je zásobníkový automat. \textit{Konfigurací} nazveme trojici $(q, w, \alpha) \in Q\times\Sigma^*\times\Gamma^*$, 
kde $q$ je aktuální stav vnitřního řízení, $w$ je dosud nezpra\-covaná část vstupního řetězce a $\alpha = Z_{i_1} Z_{i_2} \dots Z_{i_k}$ je
\mbox{obsah zásobníku}\footnote{$Z_{i_1}$ je vrchol zásobníku}. 


\subsection{Podsekce obsahující větu a odkaz}

\begin{definicia}
\label{Definice2}
Řetězec $w$ nad abecedou $\Sigma$ je přijat ZA $A$\textit{ jest\-liže $(q_0, w, Z_0)\stackrel[A]{*}{\vdash}(q_F,\epsilon,\gamma)$ pro nějaké $\gamma \in \Gamma^*$ a $q_F \in F$. 
Množinu $L(A) = \{w \mid w$ je přijat ZA $A$\} $\subseteq \Sigma^*$ nazýváme} 
jazyk přijímaný TS $M$.
\end{definicia}

Nyní si vyzkoušíme sazbu vět a důkazů opět s použitím
balíku \verb|amsthm|.

\flushbottom



\begin{veta}
\textit{Třída jazyků, které jsou přijímány ZA, odpovídá}
bezkontextovým jazykům.
\end{veta}

\begin{proof}
V důkaze vyjdeme z Definice \ref{Definice1} a \ref{Definice2}.
\end{proof}

\section{Rovnice a odkazy}
Složitější matematické formulace sázíme mimo plynulý
text. Lze umístit několik výrazů na jeden řádek, ale pak je
třeba tyto vhodně oddělit, například příkazem \verb|\quad|.

\vspace{5mm}
\noindent
$ \sqrt[i]{x^3_i} $
\quad
kde $x_i$ je $i$-té sudé číslo splňující 
\quad
$x_i^{2-x_i^{i^2}} \leq x_i^{y_i^3} $

\medskip
V rovnici (\ref{equation1}) jsou využity tři typy závorek s  různou explicitně definovanou velikostí.

\begin{eqnarray}
\label{equation1}
    x  & = &  \bigg[\Big\{\big[a + b\big] * c\Big\}^d \ominus1\bigg]^{1/2}\\
    y  & = &  \displaystyle \lim_{a\to \infty} \frac{\frac{1}{log_{10}x}}{sin^2x + cos^2x} \nonumber 
\end{eqnarray}

V této větě vidíme, jak vypadá implicitní vysázení limity
$\lim_{n \to \infty} f(n)$ v normálním odstavci textu. Podobně
je to i s dalšími symboly jako $\prod^n_{i=1} 2^i$ či $\bigcap_{A\in\mathcal{B}}A$. V pří\-padě~vzorců
$ \displaystyle \lim_{n\to\infty} f(n)$ a $\stackrel[i=1]{n}{\prod}2^i$ jsme  si vynutili méně úspornou sazbu příkazem \verb|\limits|.

\begin{eqnarray}
\label{equation2}
\int_b^a g(x)\, \mathrm{d} x & = & - \int\limits_a^b f(x) \, \mathrm{d}x\\
\overline{\overline{A \wedge B}} & \Leftrightarrow & \overline{\overline{A} \vee \overline{B}}
\label{equaiton3}
\end{eqnarray}

\section{Matice}
Pro sázení matic se velmi často používá prostředí \verb|array|
a závorky (\verb|\left|, \verb|\right|).

\begin{equation*}
\left[
\begin{array}{ccc}
       & \widehat{\beta + \gamma}  & \hat{\pi} \\
\vec a & \overleftrightarrow{AC}   &           \\
\end{array} 
\right]
= 1 \Longleftrightarrow \mathbb{Q} = \mathbf{R}
\end{equation*}

\begin{equation*}
\mathbf{A} =
\left|
\begin{array}{cccc}
a_{11} & a_{12} & \ldots & a_{1n} \\
a_{21} & a_{22} & \ldots & a_{2n} \\
\vdots & \vdots & \ddots & \vdots \\
a_{m1} & a_{m2} & \ldots & a_{mn}
\end{array}
\right|
\,=\ 
\begin{matrix}
t & u \\
v & w
\end{matrix}
\ =
tw - uv
\end{equation*}

Prostředí \verb|array| lze úspěšně využít i jinde.

\begin{equation*}
\binom{n}{k} = 
\left\{
\begin{array}{ll}
0 & \text{ pro } k < 0 \text{ nebo } k > n\\
\frac{n!}{k!(n-k)!} & \text{ pro } 0 \leq k \leq n
\end{array} 
\right.
\end{equation*}



\end{document}
